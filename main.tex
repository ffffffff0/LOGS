
\documentclass[a4,10pt]{ctexart}

\usepackage{ctex}
\usepackage[utf8]{inputenc}
\usepackage{amsfonts,amsmath,amscd,amssymb,amsthm}
\usepackage{latexsym,bm}
\usepackage{cite}
\usepackage{mathtools,mathdots,graphicx,array}
\usepackage{fancyhdr}
\usepackage{lastpage}
\usepackage{color}
\usepackage{enumitem}
\usepackage{mpdoc}
\usepackage{diagbox}
\usepackage{xcolor,tcolorbox,tikz,tkz-tab,mdframed,tikz-cd}
\usepackage{framed}
\usepackage{verbatim}
\usepackage{extarrows}
\usepackage{fontspec}
\newcommand*{\dif}{\mathop{}\!\mathrm{d}}
\newcommand*{\arsinh}{\mathop{}\!\mathrm{arsinh}}
\newcommand*{\artanh}{\mathop{}\!\mathrm{artanh}}
\newcommand*{\arcosh}{\mathop{}\!\mathrm{arcosh}}
\newcommand*{\Li}{\mathop{}\!\textrm{Li}}



\begin{document}
\pagenumbering{roman}
\title{Notes}
\author{X.W.}
\date{2023年9月}
\maketitle
\tableofcontents
\newpage
\pagenumbering{arabic}
\newpage


\section{要做的事情}

\begin{yd}{积极锻炼身体日志}{}
	\begin{itemize}
		\item 2023/9/4, 晚上跑步, 总计时间: 散步+跑步=52分钟
	\end{itemize}
\end{yd}

\begin{yd}{积极出去玩}{}
	\begin{itemize}
		\item 2023/9/2--2023/9/4, 安徽省:潜山县:天柱山.
	\end{itemize}
\end{yd}

\begin{yd}{积极学习速写}{}
	\begin{itemize}
		\item 希望有时间学学
	\end{itemize}
\end{yd}

\begin{yd}{积极阅读}{}
	\begin{itemize}
		\item 心流: 最优体验, 待阅读(2023/9/4--)
	\end{itemize}
\end{yd}


\begin{yd}{阅读 \textbf{SICP}}{}
	\begin{itemize}
		\item page 69: 层次性数据和闭包性质. 2023/9/4
	\end{itemize}
\end{yd}

\begin{yd}{阅读 \textbf{数据密集型应用系统设计}}{}
	\begin{itemize}
		\item page 71: 数据存储与检索. 2023/9/4
	\end{itemize}
\end{yd}

\begin{yd}{阅读 \textbf{计算机网络}}{}
	\begin{itemize}
		\item page 1: 计算机网络和因特网. 2023/9/4
	\end{itemize}
\end{yd}

\begin{yd}{阅读 \textbf{深入理解计算机系统}}{}
	\begin{itemize}
		\item page 23: 信息的表示和处理. 2023/9/4
	\end{itemize}
\end{yd}

\begin{yd}{吹 \textbf{Nginx 和 Kong}}{}
	\begin{itemize}
		\item 基本复现 Kong Demo. 2023/9/4
	\end{itemize}
\end{yd}

\begin{yd}{吹 \textbf{SSO 登录}}{}
	\begin{itemize}
		\item 看 IAM 代码. 2023/9/4
	\end{itemize}
\end{yd}

\begin{yd}{学习 \textbf{SpringBoot}}{}
	\begin{itemize}
		\item 看代码. 2023/9/4
	\end{itemize}
\end{yd}

\begin{yd}{吹 \textbf{OpenTelemetry}}{}
	\begin{itemize}
		\item 基本复现 NodeJS 中的使用. 2023/9/4
		\item 待看: \href{https://www.bilibili.com/video/BV1cN411a72d/?vd_source=3d6137a386838c2bcb88b1db7c993448}{K8S OpenTelemetry 分享} (2023/9/4--)
	\end{itemize}
\end{yd}

\section{Spring Note}

\section{Nginx Note}

\section{JS Note}

\section{杂记}

\section{常用环境}


上述就是一些文本了. 使用的代码是
\begin{lstlisting}{language=java}
\begin{yd}{这是一个约定}{}
              
约定的内容.
\end{yd}
\end{lstlisting}

\begin{zs}
        
这是一个注释
    
\end{zs}
    
\begin{xt}
        
这是一个习题.
    
\end{xt}
    
\begin{lt}
        
这是一个问题.
    
\end{lt}

\begin{yl}
        
这是一个引理.
    
\end{yl}

\begin{dl}{A}{}
        
这是一个定理.
    
\end{dl}
    
\begin{tl}{A}{}
        
这是一个推论.
    
\end{tl}

\begin{dy}{A}{}
        
这是一个定义.
    
\end{dy}

\begin{jl}{A}{}
        
这是一个结论.
    
\end{jl}

\begin{mt}{A}{}
        
这是一个命题.
    
\end{mt}

\begin{ti}{A}{}
        
这是一个题目.
    
\end{ti}

\begin{cx}{A}{}
        
这是一个猜想.
    
\end{cx}

\begin{zy}
        
这是注意.
    
\end{zy}

\begin{ts}
        
这是一点提示.
    
\end{ts}

\begin{lt}
        
这是一个例题.
    
\end{lt}

\begin{proof}
你还可以加一点证明. 
\end{proof}

我们注意到, 所有的数学公式将自动转换成行间公式的大小, 比如${1\over 2^k}$, $\sum_{i=0}^{998244353}i$. 

\section{起源与未来的修改计划}

起源与hkmod的模板, 添加了一些常用的标志词. 可以在mpdoc.sty里面进行更改, 相信根据注释你也会. 

使用愉快! 


\end{document}
